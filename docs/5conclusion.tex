The project has given us a solid theoretical and practical introduction to the underlying methods that make neural networks function as they do. However, we also faced issues along the way with complicated implementations.

We aimed to create a neural network completely from scratch, including implementing the gradient descent methods that are used in the training process. Through experiments with each subsequent part that was implemented, we were able to produce a network that could predict numerical values based on data generated from a polynomial, initialized with four hidden layers and four nodes in each of these layers. It used Adam to do gradient descent in its backpropagation, and did reasonably well with an MSE of around 2.5 after 10 epochs, but not as well as basic linear regression techniques. 

Following those initial tests, we constructed a network that could predict breast cancer data. From a theoretical and empirical standpoint, this should be a task a neural network is well-suited to do, and with the PyTorch implementation we saw solid performance with a neural network with \textcolor{red}{(parameters)}.

Thus, we are left with an important question: Are neural networks always worth all the tuning they require? Based on our findings, the answer is clearly \emph{no}. Our first use-case, numerical prediction, showed that OLS (implemented with a variety of gradient descent methods) was better suited than our neural network to predict polynomial data. The classification task let us see both sides: While we experienced first-hand how numerical instability and the implementation of the complex algorithms involved in machine learning can lead to hidden errors, we also saw the power of neural networks when we implemented PyTorch and were able to predict breast cancer with \textcolor{red}{(accuracy)}\% accuracy. The neural network outperformed its counterpart logistic regression, at the expense of a degree of explainability and computational time. 

In summary, we achieved our goal of exploring the creation and different usages of neural networks, and despite issues with our implementation, we saw how they can be used to predict medical data, and in this way function as a diagnostic tool in the field of medicine. In the process, we gained understanding of the nuances in neural networks, such as the cases in which a neural network may not be the best choice. It is important to tailor the methods used to the specific situation, and weigh the advantages against the risks that are presented.